%%%%%%%%%%%%%%%%%%%%%%%%%%%%%%%%%%%%%%%%%%%%%%%%%%%%%%%%%%%%%%%%%%%%%%%%%%%
%
% Análisis del impacto de la implementación de la nueva leyagroalimenticia en la competitividad del sector agrícola enel estado de Jalisco
%
%%%%%%%%%%%%%%%%%%%%%%%%%%%%%%%%%%%%%%%%%%%%%%%%%%%%%%%%%%%%%%%%%%%%%%%%%%%

% Qué tipo de documento estamos por comenzar:
\documentclass[a4paper]{article}
% Esto es para que el LaTeX sepa que el texto está en español:
\usepackage[spanish]{babel}
\selectlanguage{spanish}
% Esto es para poder escribir acentos directamente:
\usepackage[utf8]{inputenc}
\usepackage[T1]{fontenc}



%% Asigna un tamaño a la hoja y los márgenes
\usepackage[a4paper,top=3cm,bottom=2cm,left=3cm,right=3cm,marginparwidth=1.75cm]{geometry}

%% Paquetes de la AMS
\usepackage{amsmath, amsthm, amsfonts}
%% Para añadir archivos con extensión pdf, jpg, png or tif
\usepackage{graphicx}
\usepackage[colorinlistoftodos]{todonotes}
\usepackage[colorlinks=true, allcolors=blue]{hyperref}

%% Primero escribimos el título
\title{Análisis del impacto de la implementación de la nueva ley agroalimenticia en la competitividad del sector agrícola en el estado de Jalisco}
\space
\author{Llamas Díaz José Álvaro\\
  \small Universidad de Guadalajara\\
  \small CUCEA\\
  \date\small{16/octubre/2020}
}

%% Después del "preámbulo", podemos empezar el documento

\begin{document}
%% Hay que decirle que incluya el título en el documento
\maketitle

%% Aquí podemos añadir un resumen del trabajo (o del artículo en su caso) 
\begin{abstract}
En este breve escrito describiremos \textit{grosso modo} el trabajo titulado 'Análisis del impacto de la implementación de la nueva ley agroalimenticia en la competitividad del sector agrícola en el estado de Jalisco' para concer un poco mas de lo que se pretende realizar al respecto.
\end{abstract}

%% Iniciamos "secciones" que servirán como subtítulos
%% Nota que hay otra manera de añadir acentos
\section{Introducci\'on}

En México el sector primario represento, en el segundo trimestre del 2019, el 3.7\% del PIB nacional con un valor de 859,118 millones de pesos a precios de mercado; la agricultura represento en este mismo periodo el 64.6\% de los ingresos del sector primario, esto se traduce en 555,717 millones de pesos. \cite{Ced19}(CEDRSSA, 2019). \\

A nivel estatal Jalisco represento en 2018 el primer lugar a nivel nacional en términos de superficie cosechada con 1,593 millones de hectáreas seguido por Veracruz y Chiapas; se encuentra, en términos de valor productivo, en la segunda posición a nivel nacional con un valor de producción total de 66,913 millones de pesos solo por debajo de Michoacán.\\

El principal cliente externo es Estados Unidos con una partida de 29,100 millones de dólares en 2018 esto representa el 83.15\% de las exportaciones agrícolas, este sector es de alta importancia para la economía nacional \cite{OMC18}(OMC 2018).Atualmente el 93\% de los productores agricolas entran dentro de la categoria de PyMES esto basandonos en datos del \cite{Den20}DENUE (2020) y son estos ultimos quienes presentan barreras de entrada al mercado internacional debido a la carencia de mecanismos de certificacion de calidad que les permitan acceder a este mercado de mayor valor.\\

En nuestro estado, Jalisco, se aprobo en el 2019 en el congreso la nueva Ley Agroalimentaria del Estado de Jalisco, entre los objetivos prioritarios se encuentra el establecer un programa de capacitación para los pequeños productores que les permita acceder a mercados de más alto valor, esto se plasma en el Artículo 6 apartados XVI y XIX; así como aumentar la eficiencia del cultivo al reducir las pérdidas y desperdicios causados por plagas y enfermedades mediante la implementación de sistemas de calidad nacionales e internacionales. \cite{SGG19}(SGG, 2019).



\section{Propuesta de investigación}

\subsection{Hipótesis}

La implementación de la ley agroalimentaria de Jalisco impactara de manera positiva en la competitividad del sector agrícola en el estado, mediante la generación de un ambiente que favorezca el crecimiento y la competitividad por efecto de causación circular acumulativa mediante un curso de desarrollo en el que las innovaciones de proceso vendrán acompañadas de nuevas inversiones y un aumento en la productividad y la competitividad de la economía local.


\subsection{Propuesta metodológica}

Para medir la competitividad del sector agrícola en el estado de Jalisco en el presente trabajo se propone utilizar el índice de competitividad revelada aditiva (ICRA) propuesto y descrito por Hoen (2006), sumado a esto para precisar el nivel de competitividad del sector en el estado comparándolo con sus pares más cercanos en términos de valor productivo y hectáreas cosechadas, se utilizaran los siguientes indicadores.\\

Ventaja competitividad revelada aditiva (VCRA).\\

Hoen y Oosterhaven (2006), Esta variable puede ser calculada mediante la fórmula\\

\begin{equation}
    VCRA_a^i=(X_a^i/X_n^i  )-(X_a^r/X_n^r   )
\end{equation} \\
Dónde\\ 
X=Valor de las exportaciones agroalimentarias\\
a=Cualquier producto en particular\\
i=Estado de origen (Periodo de tiempo 1)\\
r=Estado de origen (Periodo de tiempo 2)\\
n=bienes comercializados menos el valor de a\\ 

Este índice arroja valores entre 1 (competitivo) y -1 (no competitivo). \\

Este índice es útil para el análisis de grandes sectores de la economía como lo es el sector agropecuario.\\


\begin{thebibliography}{X}
\bibitem{Ced19} \textsc{CEDRSSA, Centro de Estudios para el Desarrollo Rural Sustentable} y la Soberanía Alimentaria. (2019, Agosto). El Sector Agropecuario en el PIB (N.o 1). Av. Congreso de la Unión Núm. 66, Ciudad de México. 

\bibitem{Den20} \textsc{DENUE (2020)} Directorio Estadístico Nacional de Unidades Económicas. Recuperado el 12 de Julio de 2020, de DENUE: https://www.inegi.org.mx/app/mapa/denue/ 

\bibitem{OMC18} \textsc{OMC} Examen estadístico del comercio mundial, World Trade Organization (WTO), 2018, pp 138. Consultado en: https://www.wto.org/spanish/res\_s/statis\_s/wts2018\_s/wts2018\_s.pdf 

\bibitem{SGG19} \textsc{SGG} Secretaria General de Gobierno, (2019). Dirección de Publicaciones, Periódico Oficial, Tomo CCCXCVI, LEY AGROALIMENTARIA DEL ESTADO DE JALISCO

\end{thebibliography}

\end{document}